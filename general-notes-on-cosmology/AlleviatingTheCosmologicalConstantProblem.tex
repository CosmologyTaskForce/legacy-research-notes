% !TEX program = xelatex

\documentclass[12pt,a4paper]{article}

\usepackage{amsthm,amsfonts,amssymb,bm}
\usepackage[fleqn]{amsmath}
\addtolength{\textheight}{2.0cm}
\addtolength{\voffset}{-2cm}
\addtolength{\hoffset}{-1.5cm}
\addtolength{\textwidth}{4.0cm}

%\allowdisplaybreaks

%\usepackage{subeqnarray}
\usepackage{mathrsfs}
%\usepackage{color}
%\usepackage{url}
%\usepackage{ulem}
\usepackage{indentfirst}
%\usepackage{textcomp}
%\usepackage{graphics}
%\usepackage{graphicx}
%\usepackage[hang,small,bf]{caption}
%\setlength{\captionmargin}{50pt}



%%Here is the configuration for chinese. 
%\usepackage[cm-default]{fontspec}
%\usepackage{xunicode}
%\usepackage{xltxtra}
%\setmainfont{AR PL UKai CN}
%\setsansfont[BoldFont=SimHei]{KaiTi_GB2312}
%\setmonofont{NSimSun}

%\XeTeXlinebreaklocale "zh"
%\XeTeXlinebreakskip = 0pt plus 1pt


%\usepackage{tikz}
%\usetikzlibrary{mindmap,trees}
\usepackage{graphicx}
%\usepackage[hang,small,bf]{caption}
%\setlength{\captionmargin}{50pt}



\begin{document}
\title{Answer to The Cosmological Problem}
\author{A Fool}
\maketitle

\newcommand{\dd}{\mathrm d}
\newcommand{\HH}{\mathcal H}
\newcommand{\CN}{{\it Cosmologia Notebook}}
\newenvironment{eqnset}
{\begin{equation}\left \bracevert \begin{array}{l}}
{\end{array} \right. \end{equation}}

\newenvironment{eqn}
{\begin{equation}\left \bracevert \begin{array}{l}}
{\end{array} \right. \end{equation}}




Paremeters table

\begin{tabular}{ccc}
	Parameters & Value & Remark\\
	$H_0$ & $71.0 \mathrm{km\cdot s^{-1}\cdot Mpc^{-1}}$ & Current Hubble Constant \\
	c & $3\times 10^5\mathrm {km\cdot s^{-1}}$ & Speed of light in vacuum\\
\end{tabular}


How big is the universe if it is going to form a black hole? Suppose we have a universe with a radius $R$. There is a minimum radius $R_{min}$ if we do not want to live inside a black hole. 

\begin{equation}
R_{min}=\sqrt{\frac{3c^2}{8\pi G \rho}}=4.23 \mathrm{Gpc}
\end{equation}

\section{What to do?}






\hrule
\hrule
\vspace{1ex}
\begin{center}
{\Large Why This  \\ \vspace{1ex}
{\bf By Harley Flanders}}
\end{center}
\vspace{1ex}
\hrule
\vspace{5ex}






\end{document}