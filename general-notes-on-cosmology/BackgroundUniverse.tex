% !TEX program = xelatex

\documentclass{article}

\usepackage{amsmath,amsthm,amsfonts,amssymb,bm}
\addtolength{\textheight}{5.0cm}
\addtolength{\voffset}{-3.5cm}
\addtolength{\hoffset}{-2.5cm}
\addtolength{\textwidth}{4.0cm}

%\allowdisplaybreaks

\usepackage{subeqnarray}
\usepackage{mathrsfs}
\usepackage[usenames,dvipsnames]{color}
\usepackage{url}
\usepackage{ulem}
\usepackage{indentfirst}
%\usepackage{textcomp}
%\usepackage{graphics}
\usepackage{graphicx}
\usepackage[hang,small,bf]{caption}
\setlength{\captionmargin}{50pt}


%%Here is the configuration for chinese. 
%\usepackage[cm-default]{fontspec}
%\usepackage{xunicode}
%\usepackage{xltxtra}
%\setmainfont{AR PL UKai CN}
%\setsansfont[BoldFont=SimHei]{KaiTi_GB2312}
%\setmonofont{WenQuanYi Micro Hei Mono}

\XeTeXlinebreaklocale "zh"
\XeTeXlinebreakskip = 0pt plus 1pt




%includeonly{}


%\graphicspath{{Figures/}{Figures/CPL/}{Figures/DE/}{Figures/P/}{Figures/DESync/}{Figures/DESync/DEs_Sync3/}}


%\usepackage{tikz}
%\usetikzlibrary{mindmap,trees}

\begin{document}
\title{Review of Background Universe}
\author{MA Lei}
\maketitle


%%%%%%%%%%%%%%%%%%%%%%%%%%%%%%%%%%%%%%%%%%%%%%%%%%%%%%%%%
%%%%%%%%%%%%%%%%%%%%%%%%%%%%%%%%%%%%%%%%%%%%%%%%%%%%%%%
%%%%%%%%%%%%%%%%%%%%%%%%%%%%%%%%%%%%%%%%%%%%%%%%%%%%%%%%
%%%%%%%%%%%%%%%%%%%%%%%%%%%%%%%%%%%%%%%%%%%%%%%%%%%%%%%
%%%%%%%%%%%%%%%%%%%%%%%%%%%%%%%%%%%%%%%%%%%%%%%%%%%%%%%




This is a review of the backgound universe, not a starter.


The background universe is rather simple if we adopt the idea that the fundamental cosmology principle is flawless.

The background of the universe can be identified with the following quantities, i.e,,

\begin{itemize}
\item
$a$, which is the scale factor, measures the scale of the metric.

\item
$k$, is the curvature of the universe. A conformal transformation confined this to have only three possible values: $0$, $-1$, $1$.

\item
$t$, which is the age of the universe, shows the age of the universe for free falling observors.

\item
$\rho_i$, is the energy density of a constituent.

\item
$p_i$, is the pressure of a constituent.

\end{itemize}

Among these, $a$, $k$ and $t$ are the spacetime quantities while $\rho_i$ and $p_i$ are the energy momentum quantities.

To make things easier, we have to introduce several auxiliary quantities.

\begin{itemize}
\item
$H_0$, which is the Hubble function, is defined as $H=\frac{\dot a}{a} |_0$. \footnote{Though out this note, $\cdot$ stands for time derivative.}

\item
$q_0$, which is the decelarate rate today, is defined as
\begin{equation}q_0=-\frac{1}{H_0^2}\frac{\ddot a}{a}|_0    .\end{equation}
\end{itemize}

In this way, through a rather complicated computation by hand or a fast and secure computation by Mathematica of Einstein equations, we can get the freedmann equations in FRW universe, which are

\begin{eqnarray}
3(\dot a ^2 + k)/a^2 &=& 8 \pi G \rho \\
2 \ddot a/a + (\dot a^2 + k)/a^2 &=& -8 \pi G p  ,
\end{eqnarray}

or

\begin{eqnarray}
H^2 + k/a^2 &=& 8 \pi G \rho /3 \\
2 \dot H + 3H^2 + k/a^2 &=& -8 \pi G p   ,
\end{eqnarray}

or

\begin{eqnarray}
3 \ddot a &=& -4 \pi a (\rho + 3p) \\
\dot \rho + 3 (\rho + p) \dot a/a &=& 0   .
\end{eqnarray}

From these equations we can find a very useful equation,
\begin{equation}
\dot H = -4 \pi G (\rho + p) + k/a^2    .
\end{equation}


But is the background universe always so simple and clean? Never thought in that way. Let's start our story of backgound universe now.


\section{Parameters}

Previously, we have several parameters to charaterise a smooth FRW universe. The point is, that set of parameters is not so convinient.

\begin{itemize}
\item
$H_0$;
\item
$q_0$;
\item
$\Omega_0 = \frac{\rho}{3H_0^2}|_0$;
\item
$w_0$, the equation of state of the constituent in the universe.
\item
$K_0=k/a_0^2={}^3R_0/6$
\end{itemize}


\section{Famous Models}

We have a "everywhere-isotropic Robertson-Walker"\footnote{George F R Ellis, Cosmological Models} geometry, on which the famous Friedmann-Lemaitre models are defined.


FLRW universe has the metric
\begin{eqnarray}
\mathrm ds^2=-\mathrm dt^2+a^2(\mathrm dr^2f^2\mathrm d\Omega^2)
\end{eqnarray}
where $f$ can be $\sin r$,$\sinh r$ or $r$, on condition that curvature $k=+1,-1,0$.



\section{Equations}

Start from Einstein equation and diagonal energy momentum tensor, we can find two equations.
\begin{eqnarray}
3(\dot a^2+k)/a^2 &=& 8\pi G \rho \text{~~~~~~Friedmann equation}\\
2\ddot a/a+(\dot a^2 +k)/a^2 &=& -8\pi G p
\end{eqnarray}

We can combine the two equations to eliminate $k$,
\begin{equation}
3 \ddot a + 4\pi G a (\rho+3p)    . 
\end{equation}
Without suprise, this is the Raychaudhuri equation in this case, which describes gravitational attraction. This shows $\rho+3p$ is the source of gravitational attraction, or "active gravitational mass density"\footnote{Ellis etc, Cosmological Models, arxiv:gr-qc/9812046v5 page 10.}.











%%%%%%%%%%%%%%%%%%%%%%%%%%%%%%%%%%%%%%%%%%%%%%%%%%%%%%%%%
%%%%%%%%%%%%%%%%%%%%%%%%%%%%%%%%%%%%%%%%%%%%%%%%%%%%%%%
%%%%%%%%%%%%%%%%%%%%%%%%%%%%%%%%%%%%%%%%%%%%%%%%%%%%%%%%
%%%%%%%%%%%%%%%%%%%%%%%%%%%%%%%%%%%%%%%%%%%%%%%%%%%%%%%
%%%%%%%%%%%%%%%%%%%%%%%%%%%%%%%%%%%%%%%%%%%%%%%%%%%%%%%
\end{document}\