% !TEX program = xelatex

\documentclass[12pt,a4paper]{article}

\usepackage{amsthm,amsfonts,amssymb,bm}
\usepackage[fleqn]{amsmath}
\addtolength{\textheight}{2.0cm}
\addtolength{\voffset}{-2cm}
\addtolength{\hoffset}{-1.5cm}
\addtolength{\textwidth}{4.0cm}

%\allowdisplaybreaks

%\usepackage{subeqnarray}
\usepackage{mathrsfs}
%\usepackage{color}
%\usepackage{url}
%\usepackage{ulem}
\usepackage{indentfirst}
%\usepackage{textcomp}
%\usepackage{graphics}
%\usepackage{graphicx}
%\usepackage[hang,small,bf]{caption}
%\setlength{\captionmargin}{50pt}


\usepackage{enumerate}

%%Here is the configuration for chinese. 
%\usepackage[cm-default]{fontspec}
%\usepackage{xunicode}
%\usepackage{xltxtra}
%\setmainfont{AR PL UKai CN}
%\setsansfont[BoldFont=SimHei]{KaiTi_GB2312}
%\setmonofont{NSimSun}

%\XeTeXlinebreaklocale "zh"
%\XeTeXlinebreakskip = 0pt plus 1pt


%\usepackage{tikz}
%\usetikzlibrary{mindmap,trees}
\usepackage{graphicx}
%\usepackage[hang,small,bf]{caption}
%\setlength{\captionmargin}{50pt}



\begin{document}
\title{Title}
\author{MA Lei}
%\maketitle

\newcommand{\dd}{\mathrm d}
\newcommand{\HH}{\mathcal H}
\newcommand{\CN}{{\it Cosmologia Notebook}}
\newenvironment{eqnset}
{\begin{equation}\left \bracevert \begin{array}{l}}
{\end{array} \right. \end{equation}}

\newenvironment{eqn}
{\begin{equation}\left \bracevert \begin{array}{l}}
{\end{array} \right. \end{equation}}










\section{Checkpoint 2012-07-01}

Follow arXiv:1205.4688

\subsection{What do the author say?}

Transition redshift could be viewed as a new cosmological number in data fitting.

\subsection{What is transition redshift and how do it enter cosmology?}


Conventions
\begin{enumerate}[\bf\tiny{TR:Con}-1]
\item
Conventions in {\CN} 2012-A, Page ...
\end{enumerate}

\begin{enumerate}[\bf\tiny{TR-BE}-1]
\item
Deceleration parameter $q(z)$ at $z$.

\item
Transition redshift $zt$ is the redshift that make $\ddot a = 0$, thus $q(zt)=0$.

\item
Hubble function $H(z)$.

\item
Friedmann equations with $\Omega_{k0}\neq 0$. And their simplifications.

\item
Cosmography.
\begin{itemize}
\item[{\small +}]
Angular diameter distance in RSM P19.

\end{itemize}

\end{enumerate}



LCDM model equations on {\CN} 2012-A.



\subsection{Some basis concepts of statistics.}

Likelihood under a model, posterior, prior, 

Chisquare, reduced chisquare...


\subsection{How to analysis SN data?}

\subsubsection{Chisquare fitting and Chisquare distribution}

....


LogLikelihood, Under some assumptions
\begin{equation}
	-2\ln {\cal L} =\chi^2 
\end{equation}


\subsubsection{Program}


I wrote a mathematica chisquare without marginalization of $H_0$.

I can use cosmomc and getdist and mathematica to get the MCMC results. I created a table of what is the output.


What to do next?
Finish the python program or write a new mathematica program of mcmc for this simple LCDM model.
Revise cosmomc to calculate LCDM. This requires a lot of revise.


















\end{document}