% !TEX program = xelatex

\documentclass[12pt,a4paper]{article}

\usepackage{amsthm,amsfonts,amssymb,bm}
\usepackage[fleqn]{amsmath}
\addtolength{\textheight}{5.0cm}
\addtolength{\voffset}{-3.5cm}
\addtolength{\hoffset}{-2.5cm}
\addtolength{\textwidth}{4.0cm}

%\allowdisplaybreaks

%\usepackage{subeqnarray}
\usepackage{mathrsfs}
%\usepackage{color}
%\usepackage{url}
%\usepackage{ulem}
\usepackage{indentfirst}
%\usepackage{textcomp}
%\usepackage{graphics}
%\usepackage{graphicx}
%\usepackage[hang,small,bf]{caption}
%\setlength{\captionmargin}{50pt}



%%Here is the configuration for chinese. 
%\usepackage[cm-default]{fontspec}
%\usepackage{xunicode}
%\usepackage{xltxtra}
%\setmainfont{AR PL UKai CN}
%\setsansfont[BoldFont=SimHei]{KaiTi_GB2312}
%\setmonofont{NSimSun}

%\XeTeXlinebreaklocale "zh"
%\XeTeXlinebreakskip = 0pt plus 1pt


%\usepackage{tikz}
%\usetikzlibrary{mindmap,trees}
\usepackage{graphicx}
%\usepackage[hang,small,bf]{caption}
%\setlength{\captionmargin}{50pt}



\begin{document}
\title{Modifying Gravity - f(R) Gravity \\ Outline}
\author{MA Lei}
\maketitle

\pagestyle{empty}

\newcommand{\arXiv}{{\bf arXiv}}
\newcommand{\dd}{\mathrm d}
\newcommand{\HH}{\mathcal H}
\newcommand{\CN}{{\it Cosmologia Notebook - 2012-02}}
\newenvironment{eqnset}
{\begin{equation}\left \bracevert \begin{array}{l}}
{\end{array} \right. \end{equation}}

\newenvironment{eqn}
{\begin{equation}\left \bracevert \begin{array}{l}}
{\end{array} \right. \end{equation}}


%%%%%%%%%%%%%%%%%%%%%Main Content%%%%%%%%%%%%%%%%%%%%%%%%%%%%%





\section{References \& Conventions}

\paragraph{Reference:} \arXiv:astro-ph/0611321; \arXiv:1002.3868; \arXiv:astro-ph/9910176

\paragraph{Conventions} \begin{itemize}
\item
$'=\frac{\partial }{\partial \tau}$
\item
$f\equiv f(R)$
\item
$f_R\equiv \frac{\partial f}{\partial R}$
\item
$\mathcal H = a'/a$
\item
$\mathrm ds^2 = a(\tau)^2(\mathrm d\tau^2 + \gamma_{ij}\mathrm dx^i\mathrm dx^j)$ is the line element used throughout this note.
\item
$R = \frac{6a''}{a^3}$ is the scalar curvature.
\end{itemize}


\section{Main Equations}

\subsection{Jordan frame}


Start from action


\begin{eqn}
S=\frac{1}{2\kappa^2} \int \mathrm d^4x\sqrt{-g}[R + f(R)] + \int\mathrm d^4 \sqrt{-g}\mathcal L_{(M)}(x_i,g_{\mu\nu})
\end{eqn}

Variation with respect to $g_{\mu\nu}$ of this action gives \footnote{The standard procedure is really complicate. Ohanian et al has a somewhat easier method in his book. The used a special set of coordinates. $g_{\mu\nu}=\eta_{\mu\nu} + \epsilon h_{\mu\nu}$, then things become easier. Palatini method should be done with another action.}
\begin{eqn}
(1+f_R)R_{\mu\nu} - \frac{1}{2} g_{\mu\nu}(R + f) + (g_{\mu\nu}\square - \nabla_\mu\nabla_\nu)f_R = \kappa^2 T_{\mu\nu}
\end{eqn}
Or
\begin{eqn}
(1+f_R)G_{\mu\nu} + \frac{1}{2} g_{\mu\nu}(f_R R - f) + (g_{\mu\nu}\square  - \nabla_\mu\nabla_\nu)f_R = \kappa^2 T_{\mu\nu}
\end{eqn}


\paragraph{Background} Background equations are
\begin{eqnset}
(1+r_R)\mathcal H^2 + \frac{a^2}{6}f - \frac{a''}{a}f_R + \mathcal H f_R' = \frac{\kappa^2}{3}a^2\rho \\
\frac{a''}{a} - (1+f_R)\mathcal H^2 + a^2 \frac{f}{6} + \mathcal H f_R' + \frac{1}{2}f_R'' = -\frac{\kappa^2}{6}a^2(\rho + 3P)
\end{eqnset}

Unknown variables: $f$/$f_R$, $\HH$, $\rho$. $f$ should be given in a certain model.



\paragraph{Perturbations}

Metric

\begin{eqnset}
g_{00} = -a^2 (1 + 2A Y) \\
g_{0i} = -a^2 B Y_i \\
g_{ij} = a^2 (\gamma_{ij} + 2 H_L Y \gamma_{ij} + 2 H_T Y_{ij})
\end{eqnset}

$H_T$ is the anisotropic distortion of each constant time hypersurface. $H_L$ is the trace part.

E-M tensor
\begin{eqnset}
T^0_{\phantom0 0} = -\rho (1+\delta Y) \\
T^0_{\phantom0i} = (\rho + p) (v - B) Y_i \\
T^i_{\phantom i 0} = -(\rho + p)v Y^i \\
T^i_{\phantom i j} = [p\delta^i_j + \delta p\delta^i_j Y + \frac 3 2 (\rho +p)\sigma Y^i_j]=p[\gamma^i_{\phantom i j} + \pi_L \delta^i_{\phantom i j} + \pi_T Y^i_{\phantom i j}]
\end{eqnset}

$v$ is the potential of velocity, $v^i\equiv u^i/u^0=vY^i$.
Subscripts T means tranverse, or simply traceless part.
Subscripts L means Longitudinal, or trace part.


Conservation equations (calculated from $T^{\phantom {(\lambda)}\nu}_{(\lambda)\phantom\nu \mu;\nu}${\footnote{Why the same with SGR? Actually I thought the conservation law should be something from the identity that $G_{ab}^{\phantom{ab};a}=0$. This DOESN'T lead to the conclustion that $T_{ab}^{\phantom{ab};a}=0$}. However we can define some kind of density and pressure and include these in the conservatioon equation. AND this won't affect my work since I don't really use this equation here. The conservation equation are derived in the way I stated here.})
\begin{eqnset}
\delta' + (1 + w)(kv + 3H_L') + 3\HH (\frac{\delta p}{\delta \rho} - w)\delta = 0    \\
(v' - B) + \HH (1-3w)(v - B) + \frac{w'}{1+w}(v-B) - \frac{\delta p/\delta \rho}{1 + w}k^2\delta - k^2 A + \frac 2 3 k^2 \sigma =0
\end{eqnset}
Conservation equations themselves are not enough.


We have the defination of $\delta R/Y$
\begin{eqn}
\frac{\delta R}{Y} = \frac{2}{a^2}\left[ -6\frac{a''}{a}A - 3\HH A' + k^2 A +kB' + 3k\HH B + 9\HH H_L' + 3H_L'' + 2k^2(H_L + \frac{H_T}{3}) \right] 
\end{eqn}

Using the standard procedure given by Kodama et al, we can find the perturbation equations. In \CN, Page 18.

Also we can transform them into what they are in Synchronous Gauge. In \CN, Page 18, 19.




\subsection{From Jordan Frame to Einstein Frame}

\paragraph{Why this transformation}\arXiv:astro-ph/9910176 mentioned "Jordan frame formulation of a scalar-tensor theory is not viable because the energy density of the gravitational scalar field present in the theory is not bounded from below", thus violating the weak energy condition{\footnote{Weak energy condition: for timelike vector field $U^\alpha$, $\rho=T_{\alpha\beta}U^\alpha U^\beta>=0$}}. So I would like to work in Einstein frame though Einstein frame also has problems such as a violation of equivalence priciple{\footnote{equality of inertial mass and gravitational mass is equivalent to the assertion that the acceleration imparted to a body by a gravitational field is independent of the nature of the body, $m_I\cdot a = Grav\cdot m_G$.}}.

Action in Jordan frame is given by
\begin{eqn}
S=\frac{1}{2\kappa^2} \int \mathrm d^4x\sqrt{-g}[R + f(R)] + \int\mathrm d^4 \sqrt{-g}\mathcal L_{(M)}(x_i,g_{\mu\nu})
\end{eqn}

Apply a gauge transformation
\begin{eqn}
\tilde g_{\mu\nu} = \Omega^2 g_{\mu\nu} ,
\end{eqn}
we get the action in Einstein frame.
\begin{eqn}
\tilde S = \frac{1}{2\kappa^2} \int \mathrm d^4 x \sqrt{-\tilde g}\tilde R + \int \mathrm d^4 x \sqrt{-\tilde g}[ -\frac 1 2 \tilde g^{\mu\nu}(\tilde \nabla_\mu \phi)(\tilde \nabla_\nu \phi) - V(\phi)] \\
 + \int \mathrm d^4 x \sqrt{-\tilde g} e^{-2\beta \kappa \phi} \mathcal L_{(M)}(x_i, e^{-\beta \kappa \phi}\tilde g_{\mu\nu})
\end{eqn}

[ Definations of $V(\phi)$, $\beta$, $\phi$, $e^{-2\omega}$, $\Omega^2 \equiv e^{2\omega(x^\alpha)}$, in \CN, Page 23. ]

Here I write down the simplified potential{\footnote{The decompostion of $\phi(\tau)+\delta\phi(\vec x,\tau)$ is used and assume the background $\phi$ only evolves with time $\tau$.}}
$V(\phi) = \frac{Rf_R - f}{2\kappa^2 (1+f_R)^2}$. Given a explicit model, this will be determined and  may posses order 2 of $\phi$. 

Then variation gives field equation
\begin{eqn}
\tilde G_{\mu\nu} = \kappa^2 \tilde T_{\mu\nu} + \frac 12 \tilde \nabla_\mu\phi\tilde \nabla_\nu\phi + \frac 1 2 (\tilde g^{\alpha\gamma}\tilde\nabla_\alpha\phi\tilde \nabla_\gamma \phi)\tilde g_{\mu\nu} - V(\phi)\tilde g_{\mu\nu}
\end{eqn}
\begin{quote}
\begin{quote}
\begin{eqnset}
\tilde T_{\mu\nu} = (\tilde \rho + \tilde P) \tilde U_\mu\tilde U_\nu + \tilde p \tilde g_{\mu\nu}   \\
\tilde U_\mu\equiv e^{\beta\kappa\phi/2} U_\mu \\
\tilde \rho = e^{-2\beta\kappa\phi}\rho \\
\tilde p\equiv e^{-2\beta\kappa \phi}p
\end{eqnset}
\end{quote}
\end{quote}

Trace of field equation,
\begin{eqn}
G = \kappa^2 T + \frac 12 \tilde g^{\mu\nu}\tilde \nabla_\mu \phi\tilde \nabla_\nu\phi + 2 \tilde g^{\mu\nu}\tilde \nabla_\mu\phi \tilde \nabla_\nu\phi - 4V(\phi)
\end{eqn}
\begin{quotation}
\begin{quotation}
\begin{eqnset}
V_\phi\equiv \frac{\mathrm dV}{\mathrm d\phi}
\end{eqnset}
\end{quotation}
\end{quotation}


\subsection{Einstein Frame}

\paragraph{Background}

Background equations are

(From conservation equation and Field equation? I didn't derive them myself.)
\begin{eqnset}
\phi'' + 2\tilde\HH \phi' + {\tilde a}^2 V_\phi = \frac12 \kappa \beta {\tilde a}^2(\tilde \rho - 3 \tilde p)   \\
\tilde \rho' + 3\tilde \HH (\tilde \rho + \tilde p) = - \frac12 \kappa \beta \phi' (\tilde \rho - 3\tilde p)
\end{eqnset}


Field equations are
\begin{eqnset}
\tilde \HH^2 = \frac13\kappa^2 (\frac12 \phi'^2 + \tilde a^2 V(\phi) + \tilde a^2 \tilde \rho_c + \tilde a^2 \rho_\gamma)     \\
\phi'' + 2\tilde \HH \phi' + \tilde a^2 V_\phi = \frac 12 \kappa \beta \tilde a^2\tilde \rho_c     \\
\tilde \rho_c \equiv \tilde \rho_c^* e^{-\kappa\beta\phi/2}   \\
\tilde \rho_c^* = \tilde \rho^{*0}_c/\tilde a^3
\end{eqnset}

Since for radiation $\tilde p= \frac{\tilde \rho}{3}${\footnote{This is interesting because we have such transformations: $\tilde\rho\equiv e^{-2\beta\kappa\phi}\rho$ and $\tilde p \equiv e^{-2\beta\kappa\phi}p$. (Also the velocity transformation is $\tilde U_\mu\equiv e^{\beta\kappa\phi/2}U_\mu$ thus we can define a consitent E-M tensor which has the same form as in Jordan frame in Einstein frame.)}}, from conservation equations
\begin{eqn}
\tilde\HH'-\tilde\HH^2 = -\frac12 \kappa^2( \phi'^2 + \tilde\rho_c +\frac 4 3 \tilde \rho_\gamma)
\end{eqn}
\begin{eqn}
\tilde \rho_\gamma' + 4\tilde\HH\tilde \rho_\gamma =0
\end{eqn}

The actual equation for matter is 
\begin{eqn}
\tilde \rho_c' +3 \tilde\HH \tilde \rho_c = - Const\cdot \beta \kappa \phi'\tilde\rho_c
\end{eqn}


Unkown variables: $\phi$, $\tilde\HH$, $\tilde \rho_c$, $\tilde\rho_\gamma$, $\tilde p$. These two equations are just some of the complete equation system. We have to use Field equations to form a complete system.


[ Defination of $\delta_c$ and $\theta_c$, $\delta_c = \frac{\delta \tilde \rho^*_c}{\tilde \rho^*_c}$.  In \CN, Page 24.] The scalar field is decomposed into $\phi(t) + \delta\phi(\vec x,t)$.


\paragraph{Perturbations} Perturbation equations are
 
 \begin{eqnset}
 \tilde \delta_c'' + \tilde \HH \tilde \delta_c' - \frac 32 \tilde \HH ^2 (2\tilde \Omega_\gamma \tilde \delta_\gamma + \tilde\Omega_c(\tilde\delta_c - \frac 1 2\kappa \beta \delta\phi) + 2\kappa^2 \phi' \delta \phi' -\kappa^2 V_\phi \delta\phi )= 0   \\
 \delta\phi'' + 2\tilde \HH \delta\phi' + k^2 \delta\phi + \tilde a^2 V_{,\phi\phi} \delta\phi - \phi'\tilde \delta_c' - \frac 32 \frac\beta\kappa \tilde \HH^2 \tilde \Omega_c(\tilde \delta_c - \frac 12 \kappa\beta\delta\phi)  = 0   \\
 \tilde \delta_\gamma'' + \frac 13k^2\tilde \delta_\gamma -\frac 13 \tilde  \delta_c'' =0
 \end{eqnset}

And then it is possible to solve these equations since there are only 3 variables unkown, $\tilde\delta_c$, $\tilde \delta_\gamma$, $\tilde \delta\phi$.

Then we can discuss the equations in matter dominated era. In \CN, Page 26. (It is about the attractor point.)
 
\begin{eqnset}
\tilde \delta_c'' + \frac 32 \frac 1\tau \tilde \delta_c' - 3 \frac 1{\tau^2} (\tilde \delta_c -\frac 12 \kappa \beta\delta\phi) + \frac{2\kappa}{\beta}\frac 1\tau \delta\phi' = 0 \\
\delta\phi'' + 2\tilde  \HH \delta\phi' + k^2 \delta\phi - \frac{1}{\beta\kappa}\frac1\tau\tilde\delta_c'-\frac{3\beta}{\kappa}\frac{1}{\tau^2}(\tilde \delta_c - \frac 12 \kappa\beta\delta\phi) = 0
\end{eqnset}



%%%%%%%%%%%%%%%%%%%%%%%%%%%%%%%%%%%%%%
\iffalse
\vspace{5ex}
\hrule
\vspace{5ex}
\begin{equation}
S=\frac{1}{2\kappa^2} \int \mathrm d^4x\sqrt{-g}[R + f(R)] + \int\mathrm d^4 \sqrt{-g}\mathcal L_{(M)}(x_i,g_{\mu\nu})
\end{equation}
Background equations are
\begin{equation}\left \{ \begin{array}{l}
(1+r_R)\mathcal H^2 + \frac{a^2}{6}f - \frac{a''}{a}f_R + \mathcal H f_R' = \frac{\kappa^2}{3}a^2\rho \\
\frac{a''}{a} - (1+f_R)\mathcal H^2 + a^2 \frac{f}{6} + \mathcal H f_R' + \frac{1}{2}f_R'' = -\frac{\kappa^2}{6}a^2(\rho + 3P)
\end{array} \right. \end{equation}
\fi
%%%%%%%%%%%%%%%%%%%%%%%%%%%%%%%%%%%%%%%%%%%



\section{Perturbation Theory in Jordan Frame}






\end{document}