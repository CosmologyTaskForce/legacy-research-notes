


%%%%%%%%%%%%%%%%%%%%%%%%%%%%%%%%%%%%%%%%%%%%%%
%%%%%%%%%%%%%%%%%%%%%%%%%%%%%%%%%%%%%%%%%%%%%%
%%%%%%%%%%%%%%%%%%%%%%%%%%%%%%%%%%%%%%%%%%%%%%
%%%%%%%%%%%%%%%%%%%%%%%%%%%%%%%%%%%%%%%%%%%%%%
%%%%%%%%%%%%%%%%%%%%%%%%%%%%%%%%%%%%%%%%%%%%%%
%%%%%%%%%%%%%%%%%%%%%%%%%%%%%%%%%%%%%%%%%%%%%%
%%%%%%%%%%%%%%%%%%%%%%%%%%%%%%%%%%%%%%%%%%%%%%




\hrule\vspace{1pt}\hrule
\begin{center}
\mbox{{\bf Harrison Zeldovich Prescription}} \\
\vspace{0.5em}
\mbox{{Harrison(1970) \& Zeldovich(1972)}}
\end{center}
\hrule



\section{Harrison Zeldovich Prescription}
All perturbations that come into the horizon have the same amplitude.
\begin{equation}
\Delta(k_i,t_i)=\Delta(k_e,t_e)=\text{Const.} \label{HZP}
\end{equation}

\paragraph{Attention} $k=2\pi/L$ is the comoving wavenumber.







\subsection{What does that mean? \& Evolution of This Prescription.}

One kind of understanding of the assumption is
\begin{enumerate}
\item
The perturbations are generated to be the same value (comoving value).

\item
When the perturbations are outside of the horizon, their comoving measurement do not evolve.

\end{enumerate}

However, this might be wrong (and it is wrong
%footnote
\footnote{Even the largest scale potential perturbations damp to $9/10$ of their original value.}
%END footnote
). This only gives us a intuitive inspiration. The only thing we know is \ref{HZP}.



\vspace{2ex}
\begin{quote}


{\bf{Am I wrong?}}

The assumption firstly used is $\delta_0=\beta_E N^{-n}$. Or equivalently, $\delta_0\propto L^{-3n}$. Since Harrison proved that there is a throshold epoch where the initial perturbations are located, all the perturbations generated at that epoch can be discribed as $\delta_0\sim L^{-3n} \sim k^{3n}$ and $n=\frac{2}{3}$. i.e., primiordial perturbations are $\delta_0\sim L_0^{-2}\sim a(t_H)^2/L_{H-physical}^2$. [$t_0$ denotes the time when the perturbation are generated. $t_H$ is the time of horizon crossing, i.e., $2ct_H=\lambda_H$.]

\begin{quote}


During RD, $a(t)^2=2\sqrt C t$, then $\delta_H\sim a(t_H)^2/t_H^2 \sim a(t_H)^2/a(t_H)^4 \sim 1/t_H$. The amplitude of the perturbations are time dependent because the moment of horizon crossing $t_H$ are different for different perturbations. So what are the mistakes here?
\end{quote}


If $\delta_H\sim k_H^2$ remains the same for all perturbations and $P(k_H)\sim k_H$, from the defination of another measurement of perturbations
\begin{equation}\Delta(k)=\frac{k^3}{2\pi^2}P(k),
\end{equation}
we have $\Delta^2=\delta_H^2$ when the perturbation comes into the horizon ($k=\lambda_H$), which indicates \ref{HZP}.

Generally speaking, one would choose the initial power spectrum to be power law form, i.e., $P(k)\propto k^n$, or evern more simple, scale free Zeldovich spectrum, $P(k)\propto k$.
%%%%%%%%%%   Footnote     %%%%%%%%%%%
\footnote{\begin{url} {http://fisica.usac.edu.gt/public/curccaf_proc/borganihtml/node5.html}
\end{url}
}
%%%%%%%%%%%%%%%%%%%%%%%%%%%%%%%%%%
 
\end{quote}


\subsubsection{Primitive Spectrum}
It is expected that $\Delta$ is scale free.
The primitive spectrum given by inflation (or just a dimension analysis) is
\begin{equation}
P_\Phi(k)=\frac{50\pi^2}{9k^3}\bigg( \frac{k}{H_0} \bigg)^{\hat{n}-1}\delta_H^2\bigg(\frac{\Omega_m}{D_1(a=1)}\bigg)^2
\end{equation}

When $\hat{n}=1$, that is exact a scale invariant perturbation, i.e., $P(k)\sim k^{-3}$


\begin{quote}
THE PROBLEM IS, How do the orginal $\mu_0\sim L_0^{-3n}\sim k_0^2$ evolve to $P(k_0)\sim k_0$?
\begin{eqnarray}
P(k_0)\sim \mu_0^2/k_0^3 \sim k_0.
\end{eqnarray}
\end{quote}




















\iffalse


%%%%%%%%%        #Harrison(1970)#          %%%%%%%%%%%%%%
\section{Harrison(1970)}


%%%%%%%%%%    ##Notations##     %%%%%%%%%%%%%

\setcounter{section}{0}


\subsection{Notations} 

\begin{quote}
0 subscript denotes the initial value \\
* subscript denotes the threshold value \\
$\mu=\delta\rho/\rho$ \\
$L$ wavelength of perturbation  \\
$N$  number of particles in the perturbed region, in which $N\propto L_0^3$  \\
$\beta_E$ const. \\
$\lambda_p$:$\rho=3m/4\pi\lambda_p^3$ interparticle distance. $mc^2$ is the mean particle energy. \\
$\lambda_g=2GM/c^2$   \\
$\lambda_H=ca/\dot a=2ct$  Hubble distance
\end{quote}



%%%%%%%%%%    ##Why##     %%%%%%%%%%%%%
\subsection{Why $\mu_0\propto N^{-n}$}

Many would think the perturbations might originated from the thermal fluction. Then $\delta=1/N$. However, this leads to weird phenomena, such as thermal fluctuations emerge suddenly in classical cosmology.

Then they adopt $\delta_0=\beta_E N^{-n}$, whose origin can be push into era that classical cosmology can not reach by setting the initial time $t_0$ to be the threshold. And $\delta_0=\beta_E N^{-n}$ turns into the classical thermal fluctuation when $n=1/2$.

\sout{Latter we'll see that the key point of this assumption lies in the arguement of the threshold epoch, $\lambda_H=\lambda_p=\lambda_g =\lambda^*$.
}





%%%%%%%%%%%     ##Assumptions##       %%%%%%%%%%%%

\subsection{Assumptions}

\begin{itemize}




\item
\begin{enumerate}

\item Perturbed area can be treated with Friedmann model when $t<t_H$. [When $t$ is coming near $t_H$, the casual area we are studying (samller than Hubble distance) is becoming inhomogeneous.]

\item Density perturbations: $\mu_0=\beta_E N^{-n}$

\item $\mu_H$ should be larger than $\mu_m$. [Or the perturbations won't grow.]

\end{enumerate}

\item
$t_0=t^*$ [
Fluctuations in the metric at the threshold of classical cosmology. Thermal fluctions is not proper unless there are reasons for the suddenly emerge of thermal fluctuations.][Suppose we use an arbitary initial time. In classical cosmology, the origin of the fluctuations should be origined from the thermal fluctions. As a result of observations, the thermal fluctuations should suddenly emerge. This is weird in classical classical cosmology. Harrison promoted the idea of Wheeler.]


\end{itemize}




%%%%%%%%%%    ##Equations##     %%%%%%%%%%%%%
\subsection{Equations}

Early Friedmann models: $\mu=a\mu_0(t/t_0)^m$. [Radiation, $\rho a^4=const.$, $a^2=2Bt$. Or more directly, $\Delta^2(k,t)g(t)k^3/2\pi^3P(k)$, the growth factor $g(t)=g(t_0)(t/t_0)^{2/3}$ when $t_0>t_{eq}$ and $g(t)=g(t_0)$ when $t_0<t_{eq}$.]










%%%%%%%%%%%      ##Why Threshold##       %%%%%%%%%%%%
\setcounter{section}{1}
\subsection{Why threshold?} 


\begin{equation}
\frac{\lambda_H}{\lambda_p}=(\frac{\lambda_p}{\lambda_g})^{1/2}
\end{equation}

Since $\lambda_H/\lambda_p \propto (t/m)^{1/3}$, $\lambda_H/\lambda_p$ and $\lambda_p/\lambda_g$ vanishes when $t\rightarrow 0$.

Problem is, the continuous fluid models are valid under $\lambda_g \ll \lambda_p \ll \lambda_H$. Thus when $t$ is small enough, from former arguement we know there must be some critical time $2ct^*=\lambda^*$, in which $\lambda^*=\lambda_H=\lambda_p=\lambda_g$.

That means, our classical model should start at $t^*$.

%\vspace{0.5em}
\begin{center}-----------------------------------------------------------\end{center}
%\vspace{1em}
\paragraph{More}
Early universe, pressure $p=\rho/3$, scale factor $\dot a^2=1/a^2-k$, when $a$ is small, $k$ is negligible.

Then solve $\rho$ by putting $a=\sqrt{2\sqrt{C}t}$
%back into $\ddot a =-8\pi/3 \cdot a\rho$, we get 
\begin{equation}
\rho t^2=\frac{3}{32\pi G}
\end{equation}

Finally,$\dot a^2 =\frac{8\pi G}{3} \rho a^4 \frac 1 {a^2}=C\frac 1 {a^2}, \quad  t=(\lambda_p^3/4c^2\lambda_g)^{1/2}$.


%%%%%%%%%%%%%   ##n=2/3##       %%%%%%%%%%%%    
\subsection{$\bold{n=\frac{2}{3}}$ Law}
Harrison used very weak constrains. The perturbed area should stay far away from be closed and $\mu_H$ should be much larger than $\mu_m$.
\begin{itemize} 
\item 
Perturbations should not be too large because it has to ensure that the region is open on all scales ($r$), i.e., $\Gamma^2(r)<0$ for every $<r<L*$. 

\vspace{2em}

At the boundary $1-\Gamma_b^2=8\pi G \rho_0 \mu_0 R_0^2r_b^2/3=\beta_E N^{-n+2/3}t^*/t_0$ \footnote{Calculation:
%%%%%%%%%%%%%%%%%%%%%Footnote
A perturbed region,
\begin{equation}
\mathrm d s^2=\mathrm d\tau^2-S^2(\Gamma^{-2}\mathrm dr^2+r^2\mathrm d\Omega^2)
\end{equation}
\begin{equation}
\Gamma^2=1-\kappa r^2
\end{equation}


{\color{red}{Note: Here $S$ is different from the background scale factor.}}

\begin{equation}
(\mathrm dS/\mathrm d\tau)^2=(C+\delta C)/S^2-\kappa   \label{deltaC}
\end{equation}

Go on with \ref{deltaC}, 
\begin{subeqnarray}
S_0^2\kappa/C=\delta C/C
\end{subeqnarray}

Initially, $\delta C=\delta (8\pi G\rho a_0^4/3)=(8\pi G a_0^4/3 \cdot \delta \rho_0) / (8\pi G a_0^4/3 \cdot  \rho_0)=\mu_0=\beta_E N^{-n}$,  [{\em{I am a little confused here.}}]

In this way, \ref{deltaC} becomes
\begin{subeqnarray}
(\frac{\mathrm dS}{\mathrm d\tau})^2&=&C[(1+\frac{\delta C}{C})S^{-2}-\frac{\kappa}{C}] \\
&=&C[(1+\mu_0)S^{-2}-\mu_0 a_0^{-2}]
\end{subeqnarray}

%%%%%%%%%%%%%%%%%%%%%%%%%END Footnote
}, thus 
\begin{equation}
n>\frac{2}{3}+\frac{\ln{\beta_E t^*/t_0}}{\ln{N}}
\end{equation}




\vspace{2em}
\item $\mu$ has a minimium value $\mu_m\sim  10^{-2}$ at the end of radiation epoch in order to grow into galaxies.
[When the Hubble distance is crossing with the perturbation area, the perturbation is subject to damping if it is small, i.e., if the perutrbation $\mu_H$ at $t_H$ is too small.\footnote{Reference or simple illustration. Read ref. 3}] Or to show that more clearly, 
\begin{enumerate}
\item 
$\mu$ will damp after $t_H$.
\item
$\mu$ at the end of of Radiation Era should be larger than $\mu_m$.
\end{enumerate}

\vspace{2em}
We have the perturbation at the crossing
\begin{equation}\mu_H=2\beta_E N^{-n+2/3}t^*/3t_0,
\end{equation}
since
\begin{eqnarray*}
\mu_H&=&2\mu_0 t_H/3t_0=2\beta_E N^{-n} t_H/3t_0    \\
t_H&=&t^* N^{2/3}.
\end{eqnarray*}


$\mu>\mu_m$ is valid if and only if
\begin{equation}
n<\frac 2 3+\frac{\ln{(2\beta_E t^*/3\mu_m t_0)}}{\ln N}
\end{equation}

\end{itemize}

\vspace{2em}

Finally, 
\begin{equation}
0<(n-\frac{2}{3})\ln N-\ln \beta_E<\ln{(2/3\mu_m)}
\end{equation}

$N$ is suffiently large, so for a wide range of $\beta_E$, $n\sim 2/3$.\footnote{
%%%%%%%%%%%%%%%%%%%%Footnote
$\beta_E$ is a constant, Why it should be small? Or why should it be of the magnitude of $\mu_m$?
%%%%%%%%%%%%%%%%%%%END Footnote
}




\fi






%%%%%%%%%%%%%%%%%%%%%%%%%%%%%%%%%%%%%%%%%%%%%%
%%%%%%%%%%%%%%%%%%%%%%%%%%%%%%%%%%%%%%%%%%%%%%
%%%%%%%%%%%%%%%%%%%%%%%%%%%%%%%%%%%%%%%%%%%%%%
%%%%%%%%%%%%%%%%%%%%%%%%%%%%%%%%%%%%%%%%%%%%%%
%%%%%%%%%%%%%%%%%%%%%%%%%%%%%%%%%%%%%%%%%%%%%%
%%%%%%%%%%%%%%%%%%%%%%%%%%%%%%%%%%%%%%%%%%%%%%
%%%%%%%%%%%%%%%%%%%%%%%%%%%%%%%%%%%%%%%%%%%%%%
%%%%%%%%%%%%%%%%%%%%%%%%%%%%%%%%%%%%%%%%%%%%%%
%%%%%%%%%%%%%%%%%%%%%%%%%%%%%%%%%%%%%%%%%%%%%%
%%%%%%%%%%%%%%%%%%%%%%%%%%%%%%%%%%%%%%%%%%%%%%
%%%%%%%%%%%%%%%%%%%%%%%%%%%%%%%%%%%%%%%%%%%%%%
%%%%%%%%%%%%%%%%%%%%%%%%%%%%%%%%%%%%%%%%%%%%%%
%%%%%%%%%%%%%%%%%%%%%%%%%%%%%%%%%%%%%%%%%%%%%%
%%%%%%%%%%%%%%%%%%%%%%%%%%%%%%%%%%%%%%%%%%%%%%
%%%%%%%%%%%%%%%%%%%%%%%%%%%%%%%%%%%%%%%%%%%%%%

